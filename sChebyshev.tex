% Options for packages loaded elsewhere
\PassOptionsToPackage{unicode}{hyperref}
\PassOptionsToPackage{hyphens}{url}
%
\documentclass[
  12pt,
]{article}
\usepackage{amsmath,amssymb}
\usepackage{iftex}
\ifPDFTeX
  \usepackage[T1]{fontenc}
  \usepackage[utf8]{inputenc}
  \usepackage{textcomp} % provide euro and other symbols
\else % if luatex or xetex
  \usepackage{unicode-math} % this also loads fontspec
  \defaultfontfeatures{Scale=MatchLowercase}
  \defaultfontfeatures[\rmfamily]{Ligatures=TeX,Scale=1}
\fi
\usepackage{lmodern}
\ifPDFTeX\else
  % xetex/luatex font selection
  \setmainfont[]{Times New Roman}
\fi
% Use upquote if available, for straight quotes in verbatim environments
\IfFileExists{upquote.sty}{\usepackage{upquote}}{}
\IfFileExists{microtype.sty}{% use microtype if available
  \usepackage[]{microtype}
  \UseMicrotypeSet[protrusion]{basicmath} % disable protrusion for tt fonts
}{}
\makeatletter
\@ifundefined{KOMAClassName}{% if non-KOMA class
  \IfFileExists{parskip.sty}{%
    \usepackage{parskip}
  }{% else
    \setlength{\parindent}{0pt}
    \setlength{\parskip}{6pt plus 2pt minus 1pt}}
}{% if KOMA class
  \KOMAoptions{parskip=half}}
\makeatother
\usepackage{xcolor}
\usepackage[margin=1in]{geometry}
\usepackage{longtable,booktabs,array}
\usepackage{calc} % for calculating minipage widths
% Correct order of tables after \paragraph or \subparagraph
\usepackage{etoolbox}
\makeatletter
\patchcmd\longtable{\par}{\if@noskipsec\mbox{}\fi\par}{}{}
\makeatother
% Allow footnotes in longtable head/foot
\IfFileExists{footnotehyper.sty}{\usepackage{footnotehyper}}{\usepackage{footnote}}
\makesavenoteenv{longtable}
\usepackage{graphicx}
\makeatletter
\def\maxwidth{\ifdim\Gin@nat@width>\linewidth\linewidth\else\Gin@nat@width\fi}
\def\maxheight{\ifdim\Gin@nat@height>\textheight\textheight\else\Gin@nat@height\fi}
\makeatother
% Scale images if necessary, so that they will not overflow the page
% margins by default, and it is still possible to overwrite the defaults
% using explicit options in \includegraphics[width, height, ...]{}
\setkeys{Gin}{width=\maxwidth,height=\maxheight,keepaspectratio}
% Set default figure placement to htbp
\makeatletter
\def\fps@figure{htbp}
\makeatother
\setlength{\emergencystretch}{3em} % prevent overfull lines
\providecommand{\tightlist}{%
  \setlength{\itemsep}{0pt}\setlength{\parskip}{0pt}}
\setcounter{secnumdepth}{5}
\usepackage{tcolorbox}
\usepackage{amssymb}
\usepackage{yfonts}
\usepackage{bm}

\newtcolorbox{greybox}{
  colback=white,
  colframe=blue,
  coltext=black,
  boxsep=5pt,
  arc=4pt}
  
\newcommand{\ds}[4]{\sum_{{#1}=1}^{#3}\sum_{{#2}=1}^{#4}}
\newcommand{\us}[3]{\mathop{\sum\sum}_{1\leq{#2}<{#1}\leq{#3}}}

\newcommand{\ol}[1]{\overline{#1}}
\newcommand{\ul}[1]{\underline{#1}}

\newcommand{\amin}[1]{\mathop{\text{argmin}}_{#1}}
\newcommand{\amax}[1]{\mathop{\text{argmax}}_{#1}}

\newcommand{\ci}{\perp\!\!\!\perp}

\newcommand{\mc}[1]{\mathcal{#1}}
\newcommand{\mb}[1]{\mathbb{#1}}
\newcommand{\mf}[1]{\mathfrak{#1}}

\newcommand{\eps}{\epsilon}
\newcommand{\lbd}{\lambda}
\newcommand{\alp}{\alpha}
\newcommand{\df}{=:}
\newcommand{\am}[1]{\mathop{\text{argmin}}_{#1}}
\newcommand{\ls}[2]{\mathop{\sum\sum}_{#1}^{#2}}
\newcommand{\ijs}{\mathop{\sum\sum}_{1\leq i<j\leq n}}
\newcommand{\jis}{\mathop{\sum\sum}_{1\leq j<i\leq n}}
\newcommand{\sij}{\sum_{i=1}^n\sum_{j=1}^n}
	
\ifLuaTeX
  \usepackage{selnolig}  % disable illegal ligatures
\fi
\IfFileExists{bookmark.sty}{\usepackage{bookmark}}{\usepackage{hyperref}}
\IfFileExists{xurl.sty}{\usepackage{xurl}}{} % add URL line breaks if available
\urlstyle{same}
\hypersetup{
  pdftitle={Smacof Meets Chebyshev},
  pdfauthor={Jan de Leeuw - University of California Los Angeles},
  hidelinks,
  pdfcreator={LaTeX via pandoc}}

\title{Smacof Meets Chebyshev}
\author{Jan de Leeuw - University of California Los Angeles}
\date{Started October 08 2023, Version of October 09, 2023}

\begin{document}
\maketitle

{
\setcounter{tocdepth}{4}
\tableofcontents
}
\textbf{Note:} This is a working paper which will be expanded/updated frequently. All suggestions for improvement are welcome. The Rmd file, the pdf, all R files, a LaTeX version and so on are available at \url{https://github.com/deleeuw/sChebyshev}.

\section*{Notation}\label{notation}
\addcontentsline{toc}{section}{Notation}

\subsection*{Conventions}\label{conventions}
\addcontentsline{toc}{subsection}{Conventions}

Since we only work in finite dimensional vector spaces, and since our emphasis is on computation, we adopt the following conventions.

\begin{itemize}
\tightlist
\item
  A vector \emph{is} a matrix with one column.
\item
  A row-vector \emph{is} a matrix with one row.
\item
  Derivatives \emph{are} matrices.
\end{itemize}

\subsection*{Notations}\label{notations}
\addcontentsline{toc}{subsection}{Notations}

The length of vectors and the dimension of matrices will generally
be clear from the context.

\begin{itemize}
\tightlist
\item
  \(e_i\quad\) unit vector (element \(i\) is one, other elements zero).
\item
  \(e\quad\) vector with all elements one.
\item
  \(E\quad\) matrix with all elements one.
\item
  \(0\quad\) real number zero, also vector or matrix with all elements \(0\).
\item
  \(I\quad\) identity matrix.
\item
  \(J=I-\frac{ee'}{e'e}\quad\) centering matrix.
\item
  \(A\otimes B\quad\) Kronecker product of matrices \(A\) and \(B\).
\item
  \(X\oplus Y\quad\) direct sum of matrices \(X\) and \(Y\).
\item
  \(X\times Y\quad\) elementwise (Hadamard) product of matrices \(X\) and \(Y\).
\item
  \(\text{vec}(X)\quad\) matrix \(X\) to vector (columns on top of each other).
\item
  \(\text{vecr}(X)\quad\) elements below diagonal of matrix \(X\) to vector (columns on top of each other).
\item
  \(X'\quad\) transpose of matrix \(X\).
\item
  \(X^+\quad\) Moore-Penrose inverse of matrix \(X\).
\item
  \(X^{-T}\quad\) inverse of the transpose \(X'\) (and transpose of the inverse).
\item
  \(X\gtrsim Y\quad\) Loewner order of symmetric matrices (\(X-Y\) is positive semi-definite).
\item
  \(X\lesssim Y\quad\) Loewner order of symmetric matrices (\(Y-X\) is positive semi-definite).
\item
  \(:=\quad\) definition.
\item
  \(X\times Y\quad\) Cartesian product of sets \(X\) and \(Y\).
\item
  \((x,y)\quad\) is an ordered pair, i.e.~an element of \(X\times Y\).
\item
  \(x_{is}\) or \(\{X\}_{is}\quad\) element \((i,s)\) of matrix \(X\).
\item
  \(a_{\bullet s}\quad\) column \(s\) of matrix \(A\).
\item
  \(a_{i\bullet\quad}\) row \(i\) of matrix \(A\).
\item
  \([A]_{is}\quad\) submatrix \((i,s)\) of block-matrix \(A\).
\item
  \(A^{[p]}\quad\) direct sum of \(p\) copies of matrix \(A\).
\item
  \(a^{(k)}\quad\) the \(k^{th}\) element of the sequence \(\{a\}=a^{(1)},\cdots,a^{(k)},\cdots\).
\item
  \(\mathbb{R}^n\quad\) space of all vectors of length \(n\).
\item
  \(\overline{\mathbb{R}}^n\quad\) space of all centered vectors of length \(n\) (i.e.~\(x'e=0\)).
\item
  \(\mathbb{R}^{n\times p}\quad\) space of all \(n\times p\) matrices.
\item
  \(\overline{\mathbb{R}}^{n\times p}\quad\) space of all column-centered \(n\times p\) matrices (i.e.~with \(X'e=0\)).
\item
  \(f:X\Rightarrow Y\quad\) function with arguments in \(X\) and values in \(Y\).
\item
  If \(f:X\times Y\Rightarrow Z\) then \(f(\bullet,y):X\Rightarrow Z\)
  and \(f(x,\bullet):Y\Rightarrow Z\).
\item
  \(x'y\quad\) inner product in \(\mathbb{R}^n\).
\item
  \(\text{tr}\ X'Y\quad\) inner product in \(\mathbb{R}^{n\times p}\).
\item
  \(\|x\|=\sqrt{x'x}\quad\) Euclidean norm of \(x\in\mathbb{R}^n\).
\item
  \(\|X\|=\sqrt{\text{tr}\ X'X}\quad\) Euclidean norm of \(X\in\mathbb{R}^{n\times p}\).
\item
  \(\mathcal{D}f(x)\quad\) derivative of \(f\) at \(x\).
\item
  \(\mathcal{D}^2f(x)\quad\) second derivative of \(f\) at \(x\).
\item
  \(\mathcal{D}_sf(x)=\{\mathcal{D}f(x)\}_s\quad\) partial derivative with respect to \(x_s\) at \(x\).
\item
  \(\mathcal{D}_{st}f(x)=\{\mathcal{D}^2f(x)\}_{st}\quad\) second partial with respect to \(x_s\) and \(x_t\) at \(x\).
\end{itemize}

\section{Introduction}\label{introduction}

In this paper we study techniques to speed up the basic smacof iteration
\[
X^{(k+1)}=\Gamma(X^{(k)})
\]
with \(\Gamma\) the Guttman transform, by using updates of the form
\[
X^{(k+1)}=\sum_{r=1}^s\alpha_r\Gamma^r(X^{(k)})
\]
with \(\Gamma^0(X)=X\) and \(\Gamma^r(X)=\Gamma(\Gamma^{r-1}(X))\)

\subsection{Matrix Basis}\label{matrix-basis}

An important special case of DCDD imposes the constraint
\begin{equation}
X=\sum_{v=1}^r\theta_vG_v,
\label{eq:matbasis}
 \end{equation}
where the \(G_s\) are \(n\times p\) matrices. To see that this is indeed a special case
of DCDD define \(Y_s\) as the matrix collecting the \(s^{th}\) columns of all \(G_v\). Thus
there are \(r\) of these \(n\times r\) matrices \(Y_s\). Now \(\vec(X)=Y\theta\), with \(Y\)
the direct sum of the \(Y_s\), as usual, and
\begin{equation}
 \theta=\left.\begin{bmatrix}\theta\\\vdots\\\theta\end{bmatrix}\right\}r\ \text{times}.
\label{eq:thetamat}
  \end{equation}
The distinguishing DCDD characteristic in using the \emph{matrix basis} \eqref{eq:matbasis} is that all \(r\) pieces of \(\theta\) in \eqref{eq:thetamat} must be equal.

Better in configuration space

No \(V=I\)

\begin{verbatim}
One important application of the matrix basis is finding the optimal step size in
\end{verbatim}

iterative procedures, or, more generally, finding optimal weights in multistep
procedures. For the steepest descent method, for example, we choose \(G_1=X\)
and \(G_2=\nabla\sigma(X)\).

\[
    \{C_{ij}\}_{vw}=\text{tr}\ G_v'A_{ij}G_w
\]
Thus
\[
\{V_s\}_{vw}=\text{tr}\ G_v'V_0G_w
  \]
\[
    \{B_s(\theta)\}_{vw}=\text{tr}\ G_v'B_0(\theta)G_w
\]
\[\sigma(\theta):=\frac12\{1-2\theta'B(\theta)\theta+\theta'V\theta\}\]

\subsection{Matrix Based}\label{matrix-based}

\[
  X^{(k+1)}=\theta_1 X^{(k)}+\theta_2\Gamma(X^{(k)})+\theta_3\Gamma^2(X^{(k)})+\cdots+\theta_r\Gamma^{r-1}(X^{(k)})
  \]
Problem: near a fixed point \(B\) and \(V\) become almost singular (of rank one)

Chebyshev: \[\min_\theta\max_s|\theta_1+\theta_2\lambda_s+\cdots+\theta_r\lambda_s^{r-1}|\]

\begin{verbatim}
Sidi
\end{verbatim}

Generalizes relax.

\end{document}
